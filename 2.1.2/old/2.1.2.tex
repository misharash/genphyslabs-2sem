\documentclass[12pt]{article}
\frenchspacing
\usepackage[utf8x]{inputenc}
\usepackage[T2A]{fontenc}
\usepackage{amsmath}
\usepackage{amsfonts}
\usepackage{amssymb}
\usepackage[russian]{babel}
\usepackage{wrapfig}
\usepackage[left=2cm,right=2cm,top=2cm,bottom=2cm,bindingoffset=0cm]{geometry}
\author{Рашковецкий М.М., группа 526т}
\date{\today}
\title{Лабораторная работа 2.1.2\\Определение $C_p/C_v$ методом адиабатического расширения газа}
\begin{document}
	\maketitle
	
	{\parindent=1cm \hangindent=1cm \parskip=0.5cm
	{\bfseries Цель работы:} определение отношения $C_p/C_v$ для воздуха или углекислого газа по измерению давления в стеклянном сосуде. Измерения производятся сначала после адиабатического расширения газа, а затем после нагревания сосуда и газа до комнатной температуры.
	
	\hangindent=1cm
	{\bfseries Оборудование и материалы:} стеклянный сосуд; U-образный жидкостный манометр; резиновая груша; газгольдер с углекислым газом.\par}
	\section*{Краткая теория}
	
	\indent {\bfseries Экспериментальная установка.} Состоит из стеклянного сосуда A (объёмом около 20 л) с краном K, и U-образного жидкостного манометра M для измерения избыточного давления газа в сосуде. Избыточное давление в сосуде создаётся с помощью резиновой груши, подсоединённой к системе через кран $K_1$.
	
	В начале опыта в сосуде A находится исследуемый газ при комнатной температуре $T_1$ и давление $P_1$ немного больше атмосферного $P_0$. После открытия крана K, соединяющего сосуд A с атмосферой, давление и температура газа будет понижаться. Этот процесс можно считать адиабатическим, поскольку в газах равновесие по давлению устанавливается намного быстрее, чем по температуре. Поэтому
	\begin{equation}
	\label{eq:times1}
	\Delta t_P \ll \Delta t_T,
	\end{equation}
	где $\Delta t_P$ и $\Delta t_T$ обозначают времена выравнивания соответственно температуры и давления. Выполнение условия \eqref{eq:times1} зависит, конечно, и от конструкции установки, например, отверстие в кране K должно быть достаточно большим. Ниже приведены численные оценки величин $\Delta t_P$ и $\Delta t_T$ и их влияния на точность опытов. Если открыть кран K в течение такого времени $\Delta t$, что
	\begin{equation}
	\label{eq:times2}
	\Delta t_P \ll \Delta t \ll \Delta t_T,
	\end{equation}
	то процесс действительно можно считать адиабатическим.
	
	Преобразуем уравнение адиабаты
	\begin{equation}
	\label{eq:adiabate0}
	PV^\gamma=const
	\end{equation}
	с помощью уравнения Менделеева --- Клапейрона к переменным P, T. Обозначим состояние газа после повышения давления в сосуде и выравнивания температуры с комнатой <<1>>, а сразу после открытия крана и выравнивания давления с атмосферой --- <<2>>. Тогда
	\begin{equation}
	\label{eq:adiabate1}
	\left( \frac{P_1}{P_2} \right)^{\gamma-1}=\left( \frac{T_1}{T_2} \right)^\gamma.
	\end{equation}
	
	Газ адиабатически расширяется и охлаждается, причём $P_2=P_0$. Процесс происходит при переменной массе, поэтому удобнее исследовать расширение в переменных P, T (на их вычислении не сказывается то, что масса не постоянна).
	
	После того, как кран K вновь отсоединит сосуд от атмосферы, происходит медленное изохорическое нагревание газа до комнатной температуры ($T_3=T_1$) за время порядка $\Delta t_T$.
	
	Для этого процесса используем закон Гей-Люссака:
	\begin{equation}
	\label{eq:adiabate2}
	\frac{P_2}{T_2}=\frac{P_3}{T_3}=\frac{P_3}{T_1}.
	\end{equation}
	
	Подставим в \eqref{eq:adiabate2} отношение температур из \eqref{eq:adiabate1}, получим
	\begin{equation}
	\label{eq:adiabate3}
	\left( \frac{P_3}{P_2} \right)^\gamma=\left( \frac{P_1}{P_2} \right)^{\gamma-1}.
	\end{equation}
	
	Отсюда найдём $\gamma$:
	\begin{equation}
	\label{eq:adiabate_coeff}
	\gamma=\frac{\ln \frac{P_1}{P_0}}{\ln \frac{P_1}{P_3}}.
	\end{equation}
	
	В эксперименте $P_1$ и $P_3$ отличаются от $P_0$ на малую величину, измеряемую жидкостным манометром. Тогда
	$$
	\begin{array}{cc}
	P_1=P_0+\rho gh_1,&P_3=P_0+\rho gh_2.
	\end{array}
	$$
	
	В этом случае можно разложить логарифмы в ряд до первого члена и получить приближённо:
	\begin{equation}
	\label{eq:adiabate_coeff_approx}
	\gamma=\frac{\ln \left( 1+\frac{\rho gh_1}{P_0} \right)}{\ln \left( 1+\frac{\rho gh_1}{P_0} \right) - \ln \left( 1+\frac{\rho gh_2}{P_0} \right)} \approx \frac{h_1}{h_1-h_2}.
	\end{equation}
	
	Можно провести точные вычисления и определить ошибку приближения.
	
	По \eqref{eq:adiabate_coeff_approx}, для определения $\gamma$ нужно знать избыточное давление в баллоне до адиабатического расширения газа и после изохорного нагревания. Обе величины неоходимо измерять после окончания теплообмена, то есть кран K нужно закрыть через время $\Delta t$ согласно условию \eqref{eq:times2}.
	
	\noindent {\bfseries Время вытекания газа.} Оценим время выравнивания давления $\Delta t_P$, пренебрегая вязкостью газа. В данном случае малая длина трубки истечения позволяет это сделать.
	
	После открытия крана K по газу со скоростью звука пойдёт волна разрежения и через время $L/c$ достигнет дна сосуда. Тогда весь газ придёт в движение и через несколько таких интервалов истечение станет квазистационарным. Скорость истечения можно рассчитать по уравнению Бернулли для несжимаемой среды, потому что давление мало отличается от атмосферного и изменением плотности можно пренебречь:
	\begin{equation}
	\label{eq:velocity_bernoully}
	v=\sqrt{\frac{2 \left( P-P_0 \right)}{\rho_0}}.
	\end{equation}
	
	За время $dt$ из сосуда через отверстие площадью $S_r$ вытечет масса газа $\rho_0 S_r v dt$.
	
	В сосуде объёма $V_0$ давление за то же время снизится на $dP$, и масса газа при адиабатическом истечении уменьшится на
	\begin{equation}
	\label{eq:dmass_adiabate}
	dm=V_0 d\rho=\frac{V_0}{c^2} dP,
	\end{equation}
	где $c$ --- адиабатическая скорость звука:
	\begin{equation}
	\label{eq:velocity_sound_adiabate}
	c^2=\left( \frac{\partial P}{\partial \rho} \right)_S.
	\end{equation}
	
	Составив баланс вытекающей и остающейся в сосуде массы, получим дифференциальное уравнение:
	\begin{equation}
	\label{eq:gas_flowing_diff}
	\frac{dP}{\sqrt{P-P_0}}=-\frac{\sqrt{2\rho_0} S_r c^2}{V_0}dt,
	\end{equation}
	проинтегрировав, получим
	\begin{equation}
	\label{eq:gas_flowing_end}
	t_P=\frac{V_0}{S_r c}\sqrt{\frac{2 \left( P-P_0 \right)}{\gamma P_0}}.
	\end{equation}
	
	Для численной оценки используем приблизительные данные установки: диаметр сосуда 25 см, диаметр отверстия 1 см, $L = 50\,\text{см}$, $c = 340\,\text{м}/\text{с}$, $\gamma=1{,}4$, $P-P_0=10$ см водного столба. Из \eqref{eq:gas_flowing_end} найдём $t_P=0{,}1\,\text{с}$.
	
	За рассчитанное время вытекание газа из сосуда звуковая волна успеет 34 раза пройти от отверстия до дна и обратно. Этим обосновывается предположение о квазистационарности истечения.
	
	После выравнивания давления из-за инерции вытекающей струи могут происходить колебания воздуха в сосуде. Поэтому при малых временах открытия крана появляется значительная случайная погрешность за счёт колебаний и неопределённости во времени открытия крана. При больших колебания давление становятся меньше, но увеличивается теплообмен, отсюда занижение $P_3$ и $\gamma$.
	
	\noindent {\bfseries Нагревание газа от стенок сосуда.} Теперь оценим теплообмен за время выравнивания давления $\sim 0{,}5\,\text{с}$. За такое время глубина прогревания много меньше размеров сосуда, и нагревание можно считать одномерным.
	
	В уравнении теплопроводности все физические свойства среды, влияющие на процесс, отражены через один параметр --- коэффициент теплопроводности $\chi=\frac{\varkappa}{\rho c_p}$, где $\varkappa$ --- коэффициент теплопроводности, $c_p$ --- удельная теплоёмкость газа при постоянном давлении, $\rho$ --- плотность, $\left[ \chi \right] = \text{м}^2/\text{с}$.
	
	Если в начальном распределении температуры нет постоянных размерности длины, то решение может быть функцией только одного безразмерного параметра $\frac{x^2}{\chi t}$, где x --- координата, t --- время. Одному значению этого параметра соответствуют одинаковые температуры. При $\frac{x^2}{\chi t}=1$, т.е. $x=\sqrt{\chi t}$, в точном решении задачи о нагревании полупространства температура равна примерно среднему значению между постоянной температурой горячей стенки и изначально холодной среды. Это можно использовать для оценки глубины нагретого слоя.
	
	Возьмём из оценки времени вытекания $t=0{,}5\,\text{с}$. Для воздуха $c_p=0{,}99\,\text{Дж}/\left( \text{г} \cdot \text{К} \right)$, $\rho=1{,}29\,\text{кг}/\text{м}^3$, $\varkappa=2{,}50\cdot 10^{-2} \,\text{Вт}/\left( \text{м} \cdot \text{К} \right)$. Тогда $\chi=0{,}19\,\text{см}^2/\text{с}$ и $x=0{,}3\,\text{см}$.
	
	При радиусе сосуда $R=12{,}5\,\text{см}$ доля неохлаждённого воздуха составит $\frac{2\pi Rx}{\pi R^2}=0{,}05$. Избыточное давление $P_3-P_0$ и высота $h_2$ уменьшатся на такую же величину, что уменьшит $\gamma$ относительно на $\frac{h_2}{h_1-h_2}\cdot 0{,}05$ согласно \eqref{eq:adiabate_coeff_approx}. Например, при $h_1=20\,\text{см}$ и $h_2=10\,\text{см}$ $\gamma$ уменьшится на 5\%. Это много, потому что по классической теории $\gamma$ для двухатомного (1,4) и многоатомного (1,33) газов отличаются на те же 5\%. Оценка ошибки может быть уменьшена, если учесть охлаждение стенок сосуда.
	
	\noindent {\bfseries Охлаждение стенок.} Глубину охлаждения стенок сосуда можно оценить по той же формуле.
	
	Для стекла $c_p=1{,}0\,\text{кДж}/\left( \text{кг} \cdot \text{К} \right)$, $\rho=2{,}5\cdot 10^3 \,\text{кг}/\text{м}^3$, $\varkappa=1{,}0 \,\text{Вт}/\left( \text{м} \cdot \text{К} \right)$. Тогда $\chi=0{,}4\cdot 10^{-2} \,\text{см}^2/\text{с}$ и $x=0{,}045\,\text{см}$.
	
	Глубина охлаждения стекла примерно в семь раз меньше, чем воздуха. Теплоёмкость стекла на единицу объёма приблизительно на три порядка выше, чем у воздуха, значит, охлаждение стенок почти не сказывается на предыдущей оценке.
	
	Следовательно, теплопередача может оказывать существенное влияние на измеряемые величины, занижая показатель адиабаты. Поэтому рекомендуется проводить эксперимент многократно, а затем интерполировать и экстраполировать результаты.
	
	\section*{Ход работы}
	
	\section*{Обработка результатов}
\end{document}
